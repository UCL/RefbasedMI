% !TeX root = RJwrapper.tex

% article type: add-on package. 
%``short introductions to contributed R packages that are already available on
%CRAN or Bioconductor, and going beyond package vignettes in aiming to provide broader
%context and to attract a wider readership than package users. Authors need to make a strong
%case (in a motivating letter accompanying a submission) for such introductions, based for
%example on novelty in implementation and use of R, or the introduction of new data structures
%representing general architectures that invite re-use. Authors of narrower package-introduction
%articles may wish to consider alternatives such as The Journal of Open Source Software (http:
%//joss.theoj.org/) or in the life sciences, the F1000 Bioconductor (https://f1000research.
%com/channels/bioconductor) or Rpackage (https://f1000research.com/channels/rpackage)
%channels.''

% to do: 
%Create a minimal ‘filename.bib’ BibTeX file with only the entries you need, without unnecessary
%fields such as abstracts, and add \bibliography{filename} at the end of ‘filename.tex’, where
%filename is the same file name created above.

% to do at end:
% produce shortened .bib & change bibliography cmd
%Run tools::texi2pdf on ‘RJwrapper.tex’ to produce ‘RJwrapper.pdf’, using clean=FALSE to help
%in debugging if necessary.
%Iterate until ‘RJwrapper.pdf’ looks right.

%\newcommand {\E}      [1] {{\mbox{E}}\left[#1\right]}
\newcommand {\varb}   [1] {{\rm var}\left(#1\right)}
\newcommand {\cov}    [2] {{\rm cov}\left(#1,#2\right)}
\newcommand {\cind}      {\mbox{$\perp\hspace{-0.5em}\perp$}}
\newcommand {\logit}  [1] {{\rm logit}\ #1}
\newcommand {\twovec}[2] {\left( \begin{array}{c} #1 \\ #2 \end{array} \right)}
\newcommand {\N} [2] {N\left(#1,#2\right)}

\newcommand {\pret}{{\le{t}}}
\newcommand {\postt}{{>t}}
\newcommand{\bX}{\boldsymbol{X}}
\newcommand{\bY}{\boldsymbol{Y}}
\newcommand {\tmax}{T} % {t_{max}}
\newcommand{\KK}{k_1}



\title{\texttt{mimix}: Reference-based imputation of missing data}
\author{by Ian R White, Kevin McGrath, Matteo Quartagno, Suzie M Cro and James Carpenter}

\maketitle

\abstract{
In some randomised trials, outcome data are missing after participant discontinuation of allocated treatment.
Reference-based imputation is a multiple imputation technique for quantitative outcome data in this setting: it constructs the imputation model using parameters from a reference group.
We present and describe an R package, \texttt{mimix}, for performing reference-based imputation.
The package also includes a causal model variant and imputation of data missing before discontinuation of allocated treatment.
We compare its implementation with that in SAS and Stata.
}

\section{Introduction}

Missing data are a challenge for many analyses. This article tackles the specific issue of a randomised trial with a repeatedly measured quantitative outcome, where participants who discontinue their randomised treatment are not followed up thereafter and hence have missing outcome data. 
We assume that the aim is to estimate the effect of treatment on the actual outcomes of the participants, whether observed or not: this has been termed a ``de facto'' estimand \citep{Carpenter++13} or a ``treatment policy'' estimand \citep{ICHE9R1}. 
In this setting, an analysis under the commonly used missing at random (MAR) assumption would assume that the missing (post-treatment) outcomes are comparable (conditional on observed data) with the observed (on-treatment) outcomes, and would therefore estimate the effect of treatment on the outcomes of the participants if treatment was never discontinued: this has been termed a ``de jure'' estimand \citep{Carpenter++13} or a ``hypothetical'' estimand \citep{ICHE9R1}. 

% methods
\citet{Carpenter++13} proposed that missing data after discontinuation of allocated treatment could be imputed by assuming that post-discontinuation outcomes behave, in some sense, like the outcomes in a reference group. For example, if participants who have discontinued their allocated treatment are likely to be receiving similar treatment to participants who were allocated to control treatment, then the control group would be the reference group.
\citet{Carpenter++13} proposed imputing the missing data from a joint multivariate Normal (MVN) distribution for the complete (observed and unobserved) data, and proposed five ways to construct this joint distribution: ``jump to reference'' (J2R), ``copy reference'' (CR), ``copy increments in reference'' (CIR), ``last mean carried forward'' (LMCF) and ``missing at random'' (MAR). All approaches start by fitting a MVN distribution to the data from each arm. As an example, the J2R joint distribution for a participant in a specific arm takes the means for that arm up to the point of treatment discontinuation and the means for the reference arm afterwards.

The implicit assumptions behind this approach were explored by \citet{ian:RBIcausal}, who proposed a causal model in which the treatment effect after discontinuation is a specified fraction of the treatment effect at the point of discontinuation. They showed that J2R, CR and CIR are the special cases of this causal model in which the treatment effect disappears, decays or is maintained after treatment discontinuation, while LMCF and MAR are not special cases of the causal model. 

% software 
The RBI methods of \citet{Carpenter++13} were implemented in SAS in what have become known as ``the five macros'' and are available on the web page (on www.missingdata.org.uk) of the DIA working group for missing data. These macros also provide for ``delta adjustment'' in which imputed values are modified by a user-specified amount \citep{Ratitch++13}: this is useful is performing sensitivity analysis, since the RBI methods make a number of untestable assumptions \citep{ian:ZAMSTAR}.
The methods were then implemented in Stata by \citet{Cro++16}.
A practical guide to their use is given by \citet{Cro++20}.

% Aims of paper
We have implemented the RBI methods in a new R package, which includes the a full implementation of the causal model. 
The causal model was previously specified for a two-arm trial and has been extended here for a multi-arm trial.
The aim of this paper is to describe our implementation of \texttt{mimix} in R, to illustrate its use, and to describe its novel features by comparison with the implementations in SAS and Stata.

% \section{Another section}
%
%This section may contain a figure such as Figure~\ref{figure:rlogo}.
%
%\begin{figure}[htbp]
%  \centering
%  \includegraphics{Rlogo-5}
%  \caption{The logo of R.}
%  \label{figure:rlogo}
%\end{figure}

%\section{Another section}
%
%There will likely be several sections, perhaps including code snippets, such as:
%
%\begin{example}
%  x <- 1:10
%  result <- myFunction(x)
%\end{example}

%This file is only a basic article template. For full details of \emph{The R Journal} style and information on how to prepare your article for submission, see the \href{https://journal.r-project.org/share/author-guide.pdf}{Instructions for Authors}.

\section{Reference-based imputation}

\subsection{Setting} 
Following \citet{ian:RBIcausal},
we assume that quantitative outcome measurements are scheduled at baseline and at $\tmax$ occasions after randomisation. 
Let $Z$ be the random variable for the participant's randomised treatment arm, where $Z=z$ denotes an arm in which we want to impute missing values, and $Z=r$ denotes the reference arm.
Let $Y_t$ be the random variable for the participant's outcome at visit $t=0,...,\tmax$. 
It is convenient to define row vectors
$\boldsymbol{Y}_{\pret}=(Y_0,\ldots,Y_t)$ containing all outcomes up to  and including visit $t$; 
$\boldsymbol{Y}_{\postt}=(Y_{t+1},\ldots,Y_\tmax)$ containing all outcomes after visit $t$; 
and
$\boldsymbol{Y}=(Y_0,\ldots,Y_\tmax)$, containing all outcomes.
Let $D$ be the random variable for the participant's last visit prior to discontinuing treatment, so $D=0,...,\tmax$.
$Y_t$ is observable for all $t$ but only observed for $t \le D$, because we assume no off-treatment data. 
We aim to impute the unobserved values of $Y_t$ for $t>D$: we stress that these are the outcomes that existed but were unobserved, not the outcomes that would have existed if treatment had been continued.


\subsection{Reference-based imputation}

\citet{Carpenter++13}  proposed a generic MI algorithm for this setting:
\begin{enumerate}

\item For each treatment arm $z$, fit a multivariate normal model to all observed data, using a
    Bayesian approach with an improper prior and assuming MAR. The model has unstructured
    mean $\boldsymbol{\mu}_z$ and unstructured variance-covariance matrix $\boldsymbol{\Sigma}_z$.

\item For each treatment arm $z$, draw a mean vector $\boldsymbol{\mu}_z^*$ and variance-covariance matrix $\boldsymbol{\Sigma}_z^*$ from the posterior distribution.

\item For each treatment arm  $z$ and each observed treatment discontinuation visit $t$, use the drawn values to build a hypothetical  joint  distribution of 
the outcomes $\boldsymbol{Y}_{\pret}$ up to visit $t$ 
and 
the outcomes $\boldsymbol{Y}_{\postt}$ after visit  $t$, 
using one of the methods described below.
Thus a MVN distribution is built for $\boldsymbol{Y}|Z=z,D=t$.
We define the means
$\boldsymbol{\mu}_{z,\pret}$,
$\boldsymbol{\mu}_{z,\postt}$
to be the parts of $\boldsymbol{\mu}_{z}$ up to and after visit $t$ respectively.

Five methods are mainly distinguished by their choice of mean: 
\begin{enumerate}
\item Jump to reference (J2R):
mean = $\left( \boldsymbol{\mu}_{z,\pret}^*, \boldsymbol{\mu}_{r,\postt}^* \right)$.

\item Copy reference (CR):
mean = $\boldsymbol{\mu}_r^*$.

\item Copy increments in reference (CIR):
mean = $\left( \boldsymbol{\mu}_{z,\pret}^*,  \{\mu_{z,t}^* - \mu_{r,t}^* \} \boldsymbol{e}_{\tmax-t} + \boldsymbol{\mu}_{r,\postt}^*   \right)$
where $\boldsymbol{e}_p$ is a row vector $(1,\ldots,1)$ of length $p$.

\item Missing at random (MAR): mean = $\boldsymbol{\mu}_z^*$.

\item Last mean carried forward (LMCF):
mean = $\left( \boldsymbol{\mu}_{z,\pret}^*, \mu_{z,t}^*\boldsymbol{e}_{\tmax-t} \right)$.
\end{enumerate}

The variance matrices are constructed so that the regression coefficient matrix and conditional variance matrix of the potential outcomes after visit  $t$, given those before visit $t$, are taken from arm $z$ for MAR and LMCF, and from arm $r$ for J2R, CIR and CR 
\citet{Carpenter++13,ian:RBIcausal}.

\item For each treatment arm $z$ and each observed treatment discontinuation visit $t$, construct the imputation distribution of  $\boldsymbol{Y}_{\postt}$ given  $\boldsymbol{Y}_{\pret}$. 
Sample $\boldsymbol{Y}_{\postt}$ from this   conditional distribution, to create a “completed” data set.

\item Repeat steps 2-4 $m$ times, resulting in $m$ imputed data sets.

\item Analyse each imputed data set and combine the results using Rubin's rules \citep{Rubin87}.

\end{enumerate}



\subsection{Causal model}

The causal model has previously been stated for two arms only. Here we extend its statement to the multi-arm case. 
% Novelties: K depends on z as well as t in general
We define the potential outcome $Y_t(z,s)$ at visit $t$ as the outcome that would have been observable if, possibly contrary to fact, a participant received active treatment $z$ for $s$ periods only.
In particular, $Y_t(z,0)$ is the potential outcome if never treated: it is the same for all $z$ and is written $Y_t(0)$. Similarly $Y_t(z,\tmax)$ is the potential outcome if always treated with $z$.
We let 
$\mu_t(z,s) = \E[Y_t(z,s)]$, the mean of the potential outcome at visit $t$ if active treatment $z$ were received for $s$ periods only. 
We define row vectors
$\boldsymbol{Y}_{\pret}(z,s)$, $\boldsymbol{Y}_{\postt}(z,s)$, $\boldsymbol{Y}(z,s)$,
$\boldsymbol{\mu}_\pret(z,s)$, $\boldsymbol{\mu}_\postt(z,s)$ and $\boldsymbol{\mu}(z,s)$
 as before.

[Omit?:] 
The variance-covariance matrix  of the potential  outcomes is $\boldsymbol{\Sigma}(s)=\varb{\boldsymbol{Y}(s)}$ with submatrices 
$\boldsymbol{\Sigma}_{\pret\pret}(s)$,
$\boldsymbol{\Sigma}_{\postt\postt}(s)$ and
$\boldsymbol{\Sigma}_{\postt\pret}(s)$.
We define 
the matrix of regression coefficients of  potential outcomes after visit  $t$ on those up to visit  $t$
as $\boldsymbol{\beta}_t(s) = \boldsymbol{\Sigma}_{\postt\pret}(s) \boldsymbol{\Sigma}_{\pret\pret}(s)^{-1}$, and 
the residual variance of the potential outcomes after visit  $t$ given those up to visit  $t$ as
$\boldsymbol{\Omega}_t(s) = \boldsymbol{\Sigma}_{\postt\pret}(s)
\boldsymbol{\Sigma}_{\pret\pret}(s)^{-1}
\boldsymbol{\Sigma}_{\postt\pret}(s)^T$.


The key model assumption describes how the maintained effect of treatment after
discontinuation relates to the effect of treatment before discontinuation:
\begin{equation}\label{eq:del}
\boldsymbol{\mu}_{\postt}(z,t)-\boldsymbol{\mu}_{\postt}(0) =
\boldsymbol{K}_{z,t} \left\{\boldsymbol{\mu}_{\pret}(z,t)-\boldsymbol{\mu}_{\pret}(0)\right\}
\end{equation}
where $\boldsymbol{K}_{z,t}$ is a $(\tmax - t) \times (t+1)$ matrix of sensitivity parameters: it is not
identified by the data and must be specified by the user.
In practice we make a choice of $\boldsymbol{K}_{z,t}$ determined by just two parameters $(k_0,k_1)$ giving the simplified model for each $u>t$:
\begin{eqnarray}
\boldsymbol{\mu}_u(z,t) - \boldsymbol{\mu}_u(0)  & =& k_0 k_1^{v_u-v_t} \left\{ \boldsymbol{\mu}_t(z,t) - \boldsymbol{\mu}_t(0) \right\} \label{eq:mte3}
\end{eqnarray}
where  $v_t$ is the time (on a suitable scale) of visit $t$.

[At present $(k_0, k_1)$ are fixed but it may be possible to let them depend on arm. 
This makes sense for a trial with more than 2 arms. 
Would probably code them as a variable in data set so could let them vary by any individual factors e.g. reason for missing data]

Estimation involves other assumptions which make explicit the ideas of \citet{Carpenter++13}: 
that randomisation is independent of all potential outcomes; 
that step 1 of the RBI algorithm in the active arms estimates the distribution of $\boldsymbol{Y}(z,s)$,
and in the reference arm estimates the distribution of $\boldsymbol{Y}(0)$; 
that the conditional distributions follow linear regressions (which is true if the joint distribution is MVN); 
and that treatment discontinuation is unaffected by future potential partly-treated outcomes.

These give the mean of the imputation model:
\begin{eqnarray}\label{eq:causal}
&&E[\boldsymbol{Y}_{\postt}(z,t) | Z=z, \boldsymbol{Y}_{\pret}, D=t]
\nonumber\\
&&=
\boldsymbol{\beta}_{t}(z,t) \left\{\boldsymbol{Y}_{\pret} - \boldsymbol{\mu}_{\pret}(z,t)\right\}  +
\boldsymbol{K}_{z,t} \left\{\boldsymbol{\mu}_{\pret}(z,t)-\boldsymbol{\mu}_{\pret}(0)\right\} +
\boldsymbol{\mu}_{\postt}(0).
\end{eqnarray}

[Again we can handle variances... how much to say?]


\subsection{Interim missing values}

Interim missing values are values that are missing while the individual remains on treatment and are identified by being followed by later observed values. 
In the RBI algorithm and causal model, applying the constructed imputation distribution to missing values in $\boldsymbol{Y}_{\pret}$ implies imputation under MAR.
We consider MAR imputation of interim missing values to be the procedure that is most likely to be appropriate in practice, since interim missing values that occur while on treatment are unlikely to follow the same pattern as post-discontinuation missing values that occur when off treatment. 
This differs from the behaviour of the Stata package \citep{Cro++16} which by default imputes interim missing values using the same procedure as post-discontinuation values, with MAR as an option.

[ADD SOMETHING ON INDIVIDUALS WITH NO OBSERVED VALUES]

\subsection{Delta adjustment}

Delta adjustment allows imputations to differ sytematically from RBI methods and so provides a framework for performing sensitivity analysis around RBI assumptions.
We follow the approach of the five macros. 
Delta adjustment provides an increment which is added on to all values imputed after treatment discontinuation, but not to interim missing values. 
It is specified by two user-specified $T$-dimensional row vectors $\boldsymbol{a}$ and $\boldsymbol{b}$,
where $a_u$ is a shift for time $u$, 
and $b_w$ is a scaling multiplier that controls how the user-defined shift for time $u$ is applied at time $u+w$.
Formally, any imputed value $Y^*_{z,t}$ is replaced by $Y^*_{z,t} + \sum_{u=t+1}^{u=\tmax} a_u b_{u-t}$.
This means that elements of $\boldsymbol{a}$  are cumulated after treatment discontinuation. 

For example, consider an individual who discontinued treatment at the 2nd time point. 
Under the setting $b_w=1$ for all $w$, we take the vector $\boldsymbol{a}$ starting at the 3rd element and add the cumulative sums to the imputed values. 
Modifying $\boldsymbol{b}$ modifies this behaviour, so that the vector of $\boldsymbol{a}$ starting at the 3rd time point is multiplied elementwise by the vector $\boldsymbol{b}$. 
A common increment of 3 at all time points after treatment discontinuation would be achieved by setting $\boldsymbol{a}=(3,3,3,...)$ and $\boldsymbol{b}=(1,0,0,...)$.
An increment of 3 at the time point immediately after discontinuation that is halved at each subsequent time point is specified by  
$\boldsymbol{a}=(3,3,3,...)$ and $\boldsymbol{b}=(1,-1/2,-1/4,...)$.

%For example, if $b=(1,0.7, 0.5,...)$ then a participant who discontinues treatment just after visit $t$ has 
%their imputed value at time $t+1$ incremented by $a_{t+1}$, 
%their imputed value at time $t+2$ incremented by $0.7 a_{t+1} + a_{t+2}$, 
%their imputed value at time $t+3$ incremented by $0.5 a_{t+1} +0.7  a_{t+2} + a_{t+3}$, and so on. 

Delta adjustment can apply either with RBI or the causal method and is always applied after imputation.

\subsection{Covariates}

Covariates are used to improve the plausibility of MAR assumptions and to improve the precision of imputed values \citep{ian:MItutorial}. 
In \texttt{mimix}, they are included in the multivariate normal model fitting as part of the variable $Y_0$. 
[KEVIN IS THIS CORRECT? MIGHT THEY BE WRONGLY CARRIED FORWARD UNDER LMCF?]

Missing values in covariates are not allowed in \texttt{mimix}. For these missing values in the randomised setting, single imputation methods are preferred, so mean imputation or the missing indicator method should be used before calling \texttt{mimix} \citep{ian:MIinRCTs}.

%%%%%%%%%%%%%%%%%%%%%%%%%%%%%%%%%%%%%%%%%%%%%%%%%%%%%%%%%%%%%

\section{Package}

\subsection{Arguments} 

\begin{tabular}{llp{.8\linewidth}}\hline
& Option & Description \\ \hline
\multicolumn{3}{l}{\em Data options} \\
& data 	&	 Dataset in long format, i.e.\ with one record per individual per time point. \\
& covar 	&	 Covariates measured at or before baseline. Must be complete (no missing values).	\\
& depvar 	&	 Dependent (outcome) variable.	\\
& treatvar 	&	 Treatment group, coded 1,2,...	\\
& idvar 	&	 Participant identifier.	\\
& timevar 	&	 Time point for repeated measures. \\

\multicolumn{3}{l}{\em Method options} \\
& method 	&	 Reference-based imputation method: must be one of J2R, CIR, CR, LMCF, MAR, Causal.	\\
& methodvar 	&	 Alternative to method option: variable in dataset specifying the imputation method  for each individual. \\
& reference 	&	 Reference group for J2R, CIR, CR and Causal methods. \\
& referencevar 	&	 Alternative to reference option: variable in dataset specifying the reference group for each individual. \\

\multicolumn{3}{l}{\em Causal method options} \\
& K0 	&	 Causal constant for use with Causal method. The treatment effect assumed at post-discontinuation times is K0 times the treatment effect at the time of discontinuation; if K1 $\ne$ 1, this is multiplied by a decaying term. \\
& K1 	&	 Causal constant for use with Causal method. The treatment effect assumed at post-discontinuation times, implied by K0, decays exponentially by a factor K1 per time unit. \\

\multicolumn{3}{l}{\em Delta method options}\\
& delta 	&	 Optional vector $\boldsymbol{a}$ of values to add onto imputed values as described above under ``Delta adjustment''. Length must equal number of time points.\\
& dlag 	&	 Optional vector $\boldsymbol{b}$ of values to modify the values in delta. Length must equal number of time points. \\

\multicolumn{3}{l}{\em Computation options}\\
& M 	&	 Number of imputations to be created. \\
& seed 	&	 Seed value. Specify this so that a new run of the command will give the same imputed values.	\\

\multicolumn{3}{l}{\em MCMC options} \\
& prior 	&	 Prior for the variance-covariance matrix when fitting multivariate normal distributions: Jeffreys (default), uniform or ridge.	\\
& burnin 	&	 Number of burn-in iterations when fitting multivariate normal distributions.	\\
& bbetween 	&	 Number of iterations between imputed data sets when fitting multivariate normal distributions.	\\
& mle 	&	 Logical option to use maximum likelihood parameter estimates instead of MCMC draw parameters. 
[NEED TO WRITE A COMMENT ABOUT THIS -- KEVIN, OR CAN WE UNDOCUMENT IT?]	\\ \hline
\end{tabular}

\subsection{Algorithm}

The program works through the following steps.
\begin{enumerate}
\item Set up a summary table based on treatment arm and missing data pattern (i.e.\ which timepoints are unobserved).
\item Fit a multivariate normal distribution to each treatment arm using MCMC methods in package norm2 \citep{norm2}.
\item Impute all interim missing values under a MAR assumption, looping over treatments and patterns.
\item Impute post-discontinuation missing values under the user-specified assumption, looping over treatments and patterns. If methodvar and/or refvar is specified then the loop is also over teh values of these.
\item Perform delta adjustment if specified.
\item Repeat steps 2-5 M times and form into a single data frame.
\end{enumerate}
The baseline value of the outcome could be handled as an outcome or as a covariate. 
[Which do we recommend?] 

The program is based on Suzie Cro’s Stata program \texttt{mimix}.

\subsection{Outputs}

The M imputed data sets are output concatenated as one large dataset appended to
the original unimputed dataset.
[KEVIN IS IT A DATA SET, A DATA FRAME, OR OTHER?] 

The user can use the \texttt{as.mids()} function in the \texttt{mice} package to convert the output data  to \texttt{mids} data type and hence to perform analysis using Rubin's rules.


%%%%%%%%%%%%%%%%%%%%%%%%%%%%%%%%%%%%%%%%%%%%%%%%%%%%%%%%%%%%%

\section{Example: asthma trial data}

These data are taken from \citet{Cro++16} and represent 2 arms of a 5-arm trial. 
They contain 183 patients with asthma who were randomised to active treatment or placebo. 
The primary outcome is FEV at 12 weeks, and FEV is also measured at baseline and at 2, 4, and 8 weeks of follow-up.
The data contain 732 records (one per patient per follow-up visit) and 5 variables: 
id, a patient identifier; 
time, the visit number; 
treat, the randomised group (1=placebo, 2=active); 
base, the baseline FEV, to be regarded as a covariate;
and fev, the outcome variable.

The data are missing after patients discontinue their randomised treatment. A treatment policy estimand is therefore estimated using reference-based imputation. For example, J2R is implemented as follows
\begin{verbatim}
asthmaJ2R <- mimix(data='asthma', covar=base, depvar=fev, treatvar=treat, idvar=id, timevar=time,
    method='J2R', reference=1, M=5, seed=101, prior=ridge, burnin=1000)
\end{verbatim}

[KEVIN, IT SEEMS COVAR MUST BE STRING BUT DEPVAR A VARNAME, IS THIS CORRECT? NEEDS A COMMENT?]

Note that the imputed data are in wide format. 
[KEVIN / ALL - I THINK THIS IS A PROBLEM!?]
The primary outcome can be analysed using an analysis of covariance by
\begin{verbatim}
library(mice)     
fit <- with(data= as.mids(asthmaJ2R), lm(fev.12 ~ treat + base))
summary(pool(fit))    
\end{verbatim}

[GIVE RESULTS]

%%%%%%%%%%%%%%%%%%%%%%%%%%%%%%%%%%%%%%%%%%%%%%%%%%%%%%%%%%%%%

\section{Comparisons with other packages}

[TO DISCUSS -- TEXT OR TABLE? BOTH ARE INCLUDED BELOW.]

\subsection{Comparison with Stata}
This mimix is based on the Stata version and has similar functionality while adding
the causal method and delta adjustment. 
As with the Stata version the input data requires the longitudinal input data in
long format with one record per individual at each timepoint.

The program differs in how interim missing cases -- those cases which 
have a missing measurement at a timepoint previous to a later observed measurement -- are treated.        
Under Stata by default, the interim missing are treated the same as for the post-discontinuation
missing unless the interim option is explicitly used. 
Here the interims are treated as under MAR, the post-discontinuations then imputed under
the specified method. There is no interim option as there is in Stata.

Unlike Stata an option is supplied whereby the prior used in the MCMC draws can be changed from the 
default jeffreys (as in Stata) to either the ridge or uniform.

\subsection{Comparison with SAS}

Whilst this program is based on the Stata program, the latter is an adaptation of the SAS macro miwithd,
written by James Roger, subsequently updated to the Five\_Macros suite of macros.
This program uses the same approach for the delta adjustment as described in the Five\_Macros.

In comparing outputs from our program with the Five\_Macros it is to be noted that interaction between treatment and covariates
is not allowed in the SAS macros, and comparisons are only valid for example in testing the Causal model by specifically not
not using the covbytime and catcovbytime options in the Five\_Macros.

Not using these options also means that the LMCF method can be compared with either ALMCF or OLMCF in the Five\_Macros: these are the same in the mimix setting with no covariate interacting with time.

When there is no observed data (common in the acupuncture data) the first mean is used in Stata,
a warning is given in the Five\_Macros.

\subsection{summary table}

\begin{tabular}{p{.2\textwidth}p{.2\textwidth}p{.2\textwidth}p{.2\textwidth}}\hline
	&	Five macros in SAS	&	mimix in Stata	&	mimix in R	\\\hline
{\em Functionality}	&		&		&		\\
RBI	&	yes	&	yes	&	yes	\\
Causal model	&	yes	&	no	&	yes	\\
Delta method	&	yes	&	no	&	yes	\\\hline
{\em Options}	&		&		&		\\
Interim missing	&	MAR	&	RBI or MAR	&	MAR	\\
MCMC options	&	seed and thinning (=burnin)	&	seed, burnin and thinning	&	seed, burnin, thinning and prior	\\
Covariates interacting with treatment	&	never	&	always	&	always	\\
Covariates interacting with time	&	optional	&	never	&	never	\\
LMCF	&	several flavours: ALMCF, OLMCF, AFCMCF, OFCMCF	&	ALMCF only?	&	ALMCF only?	\\
No observed data	&	Warning	&	First mean is used	&	?	\\\hline
\end{tabular}



%%%%%%%%%%%%%%%%%%%%%%%%%%%%%%%%%%%%%%%%%%%%%%%%%%%%%%%%%%%%%

\section{Limitations and discussion: sketch only so far}

\begin{itemize}
\item Flexible package
\item doesn't cover other outcome types; e.g. methods have been proposed for recurrent events \citep{Keene++14}
\item The suitability of the Rubin's rules standard errors has been debated \citep{ian:Carpenter_letter,Carpenter++14,Cro++19}
\item Trials should be designed consistently with their estimand, so if the treatment-policy estimand is of interest then outcomes should be collected after treatment discontinuation. Analysis options for this setting are still in development: options include including treatment discontinuation time in the model and imputing under MAR [ref Tom at GSK]; using RBI or causal model and reserving the post-discontinuation data for model checking; or using RBI or causal model and using the post-discontinuation data to estimate model parameters such as $K_{z,t}$. 
\item We didn't code the \emph{RBI alternative} approach of \citet{ian:RBIcausal} that instead uses $\boldsymbol{\beta}_t(\tmax)$ and $\boldsymbol{\Omega}_t(\tmax)$ for all RBI methods.

\end{itemize}

%\bibliography{RJreferences}
%\bibliographystyle{sj}
\bibliography{c:/latex/library}

\address{Ian R White\\
MRC Clinical Trials Unit at UCL\\
90 High Holborn, 2nd Floor, London WC1V 6LJ\\
UK\\
ORCiD: 0000-0002-6718-7661\\
  \email{ian.white@ucl.ac.uk}}

\address{Kevin McGrath\\
MRC Clinical Trials Unit at UCL\\
90 High Holborn, 2nd Floor, London WC1V 6LJ\\
UK\\
  (ORCiD if desired)\\
  \email{kevin.mcgrath@ucl.ac.uk}}

\address{Matteo Quartagno\\
MRC Clinical Trials Unit at UCL\\
90 High Holborn, 2nd Floor, London WC1V 6LJ\\
UK\\
  (ORCiD if desired)\\
  \email{m.quartagno@ucl.ac.uk}}

\address{Suzie M Cro\\
Imperial Clinical Trials Unit \\
Imperial College London, 1st Floor, Stadium House
London, W12 7RH\\
  UK\\
  (ORCiD if desired)\\
  \email{s.cro@imperial.ac.uk}}

\address{James Carpenter\\
MRC Clinical Trials Unit at UCL\\
90 High Holborn, 2nd Floor, London WC1V 6LJ\\
UK\\
  (ORCiD if desired)\\
  \email{j.carpenter@ucl.ac.uk}}
