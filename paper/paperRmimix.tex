% !TeX root = RJwrapper.tex

% article type: add-on package. 
%``short introductions to contributed R packages that are already available on
%CRAN or Bioconductor, and going beyond package vignettes in aiming to provide broader
%context and to attract a wider readership than package users. Authors need to make a strong
%case (in a motivating letter accompanying a submission) for such introductions, based for
%example on novelty in implementation and use of R, or the introduction of new data structures
%representing general architectures that invite re-use. Authors of narrower package-introduction
%articles may wish to consider alternatives such as The Journal of Open Source Software (http:
%//joss.theoj.org/) or in the life sciences, the F1000 Bioconductor (https://f1000research.
%com/channels/bioconductor) or Rpackage (https://f1000research.com/channels/rpackage)
%channels.''

% to do: 
%Create a minimal ‘filename.bib’ BibTeX file with only the entries you need, without unnecessary
%fields such as abstracts, and add \bibliography{filename} at the end of ‘filename.tex’, where
%filename is the same file name created above.

% to do at end:
% produce shortened .bib & change bibliography cmd
%Run tools::texi2pdf on ‘RJwrapper.tex’ to produce ‘RJwrapper.pdf’, using clean=FALSE to help
%in debugging if necessary.
%Iterate until ‘RJwrapper.pdf’ looks right.

%\newcommand {\E}      [1] {{\mbox{E}}\left[#1\right]}
\newcommand {\varb}   [1] {{\rm var}\left(#1\right)}
\newcommand {\cov}    [2] {{\rm cov}\left(#1,#2\right)}
\newcommand {\cind}      {\mbox{$\perp\hspace{-0.5em}\perp$}}
\newcommand {\logit}  [1] {{\rm logit}\ #1}
\newcommand {\twovec}[2] {\left( \begin{array}{c} #1 \\ #2 \end{array} \right)}
\newcommand {\N} [2] {N\left(#1,#2\right)}

\newcommand {\pret}{{\le{t}}}
\newcommand {\postt}{{>t}}
\newcommand{\bX}{\boldsymbol{X}}
\newcommand{\bY}{\boldsymbol{Y}}
\newcommand {\tmax}{T} % {t_{max}}
\newcommand{\KK}{k_1}



\title{mimix: Reference-based imputation of missing data}
\author{by Ian R White, Kevin McGrath, Matteo Quartagno, Suzie M Cro and James Carpenter}

\maketitle

\abstract{
Reference-based imputation is a multiple imputation technique which imputes quantitative outcome data that are missing after participant discontinuation of allocated treatment in a randomised trial.
We present and describe an R package, mimix, for performing reference-based imputation, including a causal model variant, and we compare its implementation with that in SAS and Stata.
}

\section{Introduction}

Missing data are a challenge for many analyses. This article tackles the specific issue of a randomised trial with a repeatedly measured quantitative outcome, where participants who discontinue their randomised treatment are not followed up thereafter and hence have missing outcome data. 
We assume that the aim is to estimate the effect of treatment on the actual outcomes of the participants, whether observed or not: this has been termed a ``de facto'' estimand \citep{Carpenter++13} or a ``treatment policy'' estimand \citep{ICHE9R1}. 
In this setting, an analysis under the commonly used missing at random (MAR) assumption would assume that the missing (post-treatment) outcomes are comparable (conditional on observed data) with the observed (on-treatment) outcomes, and would therefore estimate the effect of treatment on the outcomes of the participants if treatment was never discontinued: this has been termed a ``de jure'' estimand \citep{Carpenter++13} or a ``hypothetical'' estimand \citep{ICHE9R1}. 

% methods
\citet{Carpenter++13} proposed that missing data after discontinuation of allocated treatment could be imputed by assuming that post-discontinuation outcomes behave, in some sense, like the outcomes in a reference group. For example, if participants who have discontinued their allocated treatment are likely to be receiving similar treatment to participants who were allocated to control treatment, then the control group would be the reference group.
\citet{Carpenter++13} proposed imputing the missing data from a joint multivariate Normal (MVN) distribution for the complete (observed and unobserved) data, and proposed five ways to construct this joint distribution: ``jump to reference'' (J2R), ``copy reference'' (CR), ``copy increments in reference'' (CIR), ``last mean carried forward'' (LMCF) and ``missing at random'' (MAR). All approaches start by fitting a MVN distribution to the data from each arm. As an example, the J2R joint distribution for a participant in a specific arm takes the means for that arm up to the point of treatment discontinuation and the means for the reference arm afterwards.

The implicit assumptions behind this approach were explored by \citet{ian:RBIcausal}, who proposed a causal model in which the treatment effect after discontinuation is a specified multiple of the treatment effect at the point of discontinuation. They showed that J2R, CR and CIR are the special cases of this causal model in which the treatment effect disappears, decays or is maintained after treatment discontinuation, while LMCF and MAR are not special cases of the causal model. 

% software 
The RBI methods of \citet{Carpenter++13} were implemented in SAS in what have become known as ``the five macros'' and are available on the web page (on www.missingdata.org.uk) of the DIA working group for missing data. These macros also provide for ``delta-adjustment'' in which imputed values are modified by a user-specified amount \citep{Ratitch++13}: this is useful is performing sensitivity analysis, since the RBI methods make a number of untestable assumptions \citep{ian:ZAMSTAR}.
The methods were then implemented in Stata by \citet{Cro++16}.

% Aims of paper
We have implemented the RBI methods in a new R package, which includes the a full implementation of the causal model. 
The causal model was previously specified for a two-arm trial and has been extended here for a multi-arm trial.
The aim of this paper is to describe our implementation of mimix in R, to illustrate its use, and to describe its novel features by comparison with the implementations in SAS and Stata.

% \section{Another section}
%
%This section may contain a figure such as Figure~\ref{figure:rlogo}.
%
%\begin{figure}[htbp]
%  \centering
%  \includegraphics{Rlogo-5}
%  \caption{The logo of R.}
%  \label{figure:rlogo}
%\end{figure}

%\section{Another section}
%
%There will likely be several sections, perhaps including code snippets, such as:
%
%\begin{example}
%  x <- 1:10
%  result <- myFunction(x)
%\end{example}

%This file is only a basic article template. For full details of \emph{The R Journal} style and information on how to prepare your article for submission, see the \href{https://journal.r-project.org/share/author-guide.pdf}{Instructions for Authors}.

\section{Reference-based imputation}

\subsection{Setting} [nicked from \citet{ian:RBIcausal}]
We assume that quantitative outcome measurements are scheduled at baseline and at $\tmax$ occasions after randomisation. 
Let $Z$ be the random variable for the participant's randomised treatment arm, let $Z=z$ denote an arm in which we want to impute missing values, and $Z=r$ denote the reference arm.
Let $Y_t$ be the random variable for the participant's outcome at visit $t=0,...,\tmax$. 
It is convenient to define 
$\boldsymbol{Y}_{\pret}=(Y_0,\ldots,Y_t)$, the vector of all outcomes up to  and including visit $t$; 
$\boldsymbol{Y}_{\postt}=(Y_{t+1},\ldots,Y_\tmax)$, the vector of all outcomes after visit $t$; 
and
$\boldsymbol{Y}=(Y_0,\ldots,Y_\tmax)$, the vector of all outcomes.
Let $D$ be the random variable for the participant's last visit prior to discontinuing treatment, so $D=0,...,\tmax$.
$Y_t$ is observable for all $t$ but only observed for $t \le D$, because we assume no off-treatment data. 
We aim to impute the unobserved values of $Y_t$ for $t>D$: we stress that these are the outcomes that existed but were unobserved, not the outcomes that would have existed if treatment had been continued.


\subsection{Reference-based imputation}

\citet{Carpenter++13}  proposed a generic MI algorithm for this setting:
\begin{enumerate}

\item For each treatment arm $z$, fit a multivariate normal model to all observed data, using a
    Bayesian approach with an improper prior and assuming MAR. The model has unstructured
    mean $\boldsymbol{\mu}_z$ and unstructured variance-covariance matrix $\boldsymbol{\Sigma}_z$.

\item For each treatment arm $z$, draw a mean vector $\boldsymbol{\mu}_z^*$ and variance-covariance matrix from the
    posterior distribution $\boldsymbol{\Sigma}_z^*$.

\item For each treatment arm  $z$ and each possible treatment discontinuation visit $t$, use the drawn values to build a hypothetical  joint  distribution of 
the outcomes $\boldsymbol{Y}_{\pret}$ up to time $t$ 
and 
the outcomes $\boldsymbol{Y}_{\postt}$ after time  $t$, 
using one of the methods described below.
Thus a MVN distribution is built for $\boldsymbol{Y}|Z=z,D=t$.
Five methods are mainly distinguished by their choice of mean: 
\begin{enumerate}
\item Jump to reference (J2R):
mean = $\left( \boldsymbol{\mu}_{z,\pret}^*, \boldsymbol{\mu}_{r,\postt}^* \right)$.

\item Copy reference (CR):
mean = $\boldsymbol{\mu}_r^*$.

\item Copy increments in reference (CIR):
mean = $\left( \boldsymbol{\mu}_{z,\pret}^*,  \{\mu_{z,t}^* - \mu_{r,t}^* \} \boldsymbol{e}_{\tmax-t} + \boldsymbol{\mu}_{r,\postt}^*   \right)$
where $\boldsymbol{e}_p$ is a row vector $(1,\ldots,1)$ of length $p$.

\item Missing at random (MAR): mean = $\boldsymbol{\mu}_z^*$.

\item Last mean carried forward (LMCF):
mean = $\left( \boldsymbol{\mu}_{z,\pret}^*, \mu_{z,t}^*\boldsymbol{e}_{\tmax-t} \right)$.
\end{enumerate}

CRK proposed corresponding variance matrices. Following \cite{ian:RBIcausal}, we describe only the regression coefficient matrix and conditional variance matrix of the potential outcomes after visit  $t$ given those before visit $t$. 
CRK set these to be taken from arm $z$ for MAR and LMCF, 
and from arm $r$ for J2R, CIR and CR. 

[KEVIN: An approach that we call \emph{RBI alternative} instead uses
$\boldsymbol{\beta}_t(\tmax)$ and $\boldsymbol{\Omega}_t(\tmax)$ for all RBI methods: did we code this???]

\item For each treatment arm $z$ and each observed treatment discontinuation visit $t$, construct the imputation distribution of  $\boldsymbol{Y}_{\postt}$ given  $\boldsymbol{Y}_{\pret}$. 
Sample $\boldsymbol{Y}_{\postt}$ from this   conditional distribution, to create a “completed” data set.

\item Repeat steps 2-4 $m$ times, resulting in $m$ imputed data sets.

\item Analyse each imputed data set and combine the results using Rubin's rules \citep{Rubin87}.

\end{enumerate}



\subsection{Causal model}

The causal model has previously been stated for two arms only. Here we extend its statement to the multi-arm case. 
% Novelties: K depends on z as well as t in general
We define the potential outcome $Y_t(z,s)$ at visit $t$ as the outcome that would have been observable if, possibly contrary to fact, a participant received active treatment $z$ for $s$ periods only.
In particular, $Y_t(z,0)$ is the potential outcome if never treated: it is the same for all $z$ and is written $Y_t(0)$. Similarly $Y_t(z,\tmax)$ is the potential outcome if always treated with $z$.
We define 
$\boldsymbol{Y}_{\pret}(z,s)$, $\boldsymbol{Y}_{\postt}(z,s)$ and $\boldsymbol{Y}(z,s)$ as before.
We let 
$\mu_t(z,s) = \E[Y_t(z,s)]$, the mean of the potential outcome at visit $t$ if active treatment $z$ were received for $s$ periods only. Similarly we define 
$\boldsymbol{\mu}_\pret(z,s)$, $\boldsymbol{\mu}_\postt(z,s)$ and $\boldsymbol{\mu}(z,s)$.

[Omit?:] 
The variance-covariance matrix  of the potential  outcomes is $\boldsymbol{\Sigma}(s)=\varb{\boldsymbol{Y}(s)}$ with submatrices 
$\boldsymbol{\Sigma}_{\pret\pret}(s)$,
$\boldsymbol{\Sigma}_{\postt\postt}(s)$ and
$\boldsymbol{\Sigma}_{\postt\pret}(s)$.
We define 
the matrix of regression coefficients of  potential outcomes after visit  $t$ on those up to visit  $t$
as $\boldsymbol{\beta}_t(s) = \boldsymbol{\Sigma}_{\postt\pret}(s) \boldsymbol{\Sigma}_{\pret\pret}(s)^{-1}$, and 
the residual variance of the potential outcomes after visit  $t$ given those up to visit  $t$ as
$\boldsymbol{\Omega}_t(s) = \boldsymbol{\Sigma}_{\postt\pret}(s)
\boldsymbol{\Sigma}_{\pret\pret}(s)^{-1}
\boldsymbol{\Sigma}_{\postt\pret}(s)^T$.


The key model assumption describes how the maintained effect of treatment after
discontinuation relates to the effect of treatment before discontinuation:
\begin{equation}\label{eq:del}
\boldsymbol{\mu}_{\postt}(z,t)-\boldsymbol{\mu}_{\postt}(0) =
\boldsymbol{K}_{z,t} \left\{\boldsymbol{\mu}_{\pret}(z,t)-\boldsymbol{\mu}_{\pret}(0)\right\}
\end{equation}
where $\boldsymbol{K}_{z,t}$ is a $(\tmax - t) \times (t+1)$ matrix of sensitivity parameters: it is not
identified by the data and must be specified by the user.
In practice we make a choice of $\boldsymbol{K}_{z,t}$ determined by just two parameters $(k_0,k_1)$ giving the simplified model for each $u>t$:
\begin{eqnarray}
\boldsymbol{\mu}_u(z,t) - \boldsymbol{\mu}_u(0)  & =& k_0 k_1^{v_u-v_t} \left\{ \boldsymbol{\mu}_t(z,t) - \boldsymbol{\mu}_t(0) \right\} \label{eq:mte3}
\end{eqnarray}
where  $v_t$ is the time (on a suitable scale) of visit $t$.

[KEVIN - can we let $(k_0, k_1)$ depend on arm?]

Estimation involves other assumptions which make explicit the ideas of \citet{Carpenter++13}: 
that randomisation is independent of all potential outcomes; 
that step 1 of the RBI algorithm in the active arms estimates the distribution of $\boldsymbol{Y}(z,s)$,
and in the reference arm estimates the distribution of $\boldsymbol{Y}(0)$; 
that the conditional distributions follow linear regressions (which is true if the joint distribution is MVN); 
and that treatment discontinuation is unaffected by future potential partly-treated outcomes.

These give the mean of the imputation model:
\begin{eqnarray}\label{eq:causal}
&&E[\boldsymbol{Y}_{\postt}(z,t) | Z=z, \boldsymbol{Y}_{\pret}, D=t]
\nonumber\\
&&=
\boldsymbol{\beta}_{t}(z,t) \left\{\boldsymbol{Y}_{\pret} - \boldsymbol{\mu}_{\pret}(z,t)\right\}  +
\boldsymbol{K}_{z,t} \left\{\boldsymbol{\mu}_{\pret}(z,t)-\boldsymbol{\mu}_{\pret}(0)\right\} +
\boldsymbol{\mu}_{\postt}(0).
\end{eqnarray}

[Again we can handle variances... how much to say?]


\subsection{Delta adjustment}

We follow the approach of the five macros. Any imputed value $Y^*_{z,t}$ is replaced by $Y^*_{z,t} + \sum_{u=t+1}^{u=\tmax} a_u b_{u-t}$ 
where $a_u$ is a user-specified shift for time $u$ and $b_w$ is a user-specified scaling multiplier that controls how the user-defined shift for time $u$ is applied to all times $\ge u$. For example, if $b=(1,0.7, 0.5,...)$ then a participant who discontinues treatment just after visit $t$ has 
their imputed value at time $t+1$ incremented by $a_{t+1}$, 
their imputed value at time $t+2$ incremented by $0.7 a_{t+1} + a_{t+2}$, 
their imputed value at time $t+3$ incremented by $0.5 a_{t+1} +0.7  a_{t+2} + a_{t+3}$, and so on. 
Delta adjustment can apply either with RBI or the causal method and is always applied after imputation.

\subsection{Interim missing values and covariates}

[Kevin - to discuss]

%%%%%%%%%%%%%%%%%%%%%%%%%%%%%%%%%%%%%%%%%%%%%%%%%%%%%%%%%%%%%

\section{Package}

KEVIN TO DRAFT - INITIALLY TAKE FROM HELP.

\subsection{Arguments}
\subsection{Outputs}

%%%%%%%%%%%%%%%%%%%%%%%%%%%%%%%%%%%%%%%%%%%%%%%%%%%%%%%%%%%%%

\section{Examples}

KEVIN TO DRAFT. Which data to use?

%%%%%%%%%%%%%%%%%%%%%%%%%%%%%%%%%%%%%%%%%%%%%%%%%%%%%%%%%%%%%

\section{Comparisons with other packages}

KEVIN TO DRAFT -- THESE ARE JUST MY NOTES. Would a table be suitable?

\begin{enumerate}
\item flavours of LMCF in 5 macros
\item handling participants with no observed outcomes
\item how baseline covariates are modelled: SAS, can interact with time; Stata, R, can't
\item handling of interim missing values
\item anything to say about algorithm differences? prior choice?
\end{enumerate}

%%%%%%%%%%%%%%%%%%%%%%%%%%%%%%%%%%%%%%%%%%%%%%%%%%%%%%%%%%%%%

\section{Limitations and discussion}

\begin{itemize}
\item flexible package
\item doesn't cover other outcome types; e.g. methods have been proposed for recurrent events \citep{Keene++14}
\item trials should be designed consistently with their estimand, so if the treatment-policy estimand is of interest then outcomes should be collected after treatment discontinuation. Analysis options for this setting are still in development: options include including treatment discontinuation time in the model and imputing under MAR [ref Tom at GSK]; using RBI or causal model and reserving the post-discontinuation data for model checking; or using RBI or causal model and using the post-discontinuation data to estimate model parameters such as $K_{z,t}$. 
\end{itemize}

%\bibliography{RJreferences}
%\bibliographystyle{sj}
\bibliography{c:/latex/library}

\address{Ian R White\\
MRC Clinical Trials Unit at UCL\\
90 High Holborn, 2nd Floor, London WC1V 6LJ\\
UK\\
ORCiD: 0000-0002-6718-7661\\
  \email{ian.white@ucl.ac.uk}}

\address{Kevin McGrath\\
MRC Clinical Trials Unit at UCL\\
90 High Holborn, 2nd Floor, London WC1V 6LJ\\
UK\\
  (ORCiD if desired)\\
  \email{kevin.mcgrath@ucl.ac.uk}}

\address{Matteo Quartagno\\
MRC Clinical Trials Unit at UCL\\
90 High Holborn, 2nd Floor, London WC1V 6LJ\\
UK\\
  (ORCiD if desired)\\
  \email{m.quartagno@ucl.ac.uk}}

\address{Suzie M Cro\\
Imperial Clinical Trials Unit \\
Imperial College London, 1st Floor, Stadium House
London, W12 7RH\\
  UK\\
  (ORCiD if desired)\\
  \email{s.cro@imperial.ac.uk}}

\address{James Carpenter\\
MRC Clinical Trials Unit at UCL\\
90 High Holborn, 2nd Floor, London WC1V 6LJ\\
UK\\
  (ORCiD if desired)\\
  \email{j.carpenter@ucl.ac.uk}}
